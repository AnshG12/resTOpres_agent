\documentclass{beamer}
\usetheme{Madrid}
\usecolortheme{default}

\usepackage[utf8]{inputenc}
\usepackage{amsmath}
\usepackage{amssymb}
\usepackage{graphicx}
\usepackage{cleveref}
\usepackage{booktabs}

\title{Presentation}
\author{}
\institute{}
\date{\today}

\begin{document}

\begin{frame}
  \titlepage
\end{frame}


\begin{frame}
\frametitle{Overview}

\begin{itemize}
  \item Top 3 Contributions
  \item \textbf{Introduces a Scalable Interactive Oversight Method} – A novel framework for human-AI collaboration that scales oversight throughout the interaction process.
  \item \textbf{Defines Practical User Feedback Signals} – Incorporates clear user response options like \texttt{DontCare} and \texttt{DontKnow} to handle uncertainty and scope efficiently.
  \item \textbf{Establishes a Complete, Terminable Interaction Loop} – Implements a structured iterative process that guarantees resolution by visiting all nodes until oversight is achieved.
\end{itemize}

\end{frame}

\begin{frame}
        \frametitle{Preliminary \& Problem Setup}

    \begin{itemize}
      \item Our method empowers non-experts to steer advanced LLMs, bridging the gap to professional-quality outputs.
  \item We tackle the core challenge of evaluating alignment when user intent is latent and not directly observable.
  \item The research setup operationalizes the theoretical "sandwich" protocol to enable rigorous, structured assessment.
  \item This protocol provides a concrete framework to measure how well LLM outputs match the user's true, underlying goals.
    \end{itemize}

    \end{frame}
    
\begin{frame}
        \frametitle{Method: Scalable Interactive Oversight}

    \begin{itemize}
      \item Introduces an interaction agent for scalable oversight through a decomposition-interaction loop.
  \item Users can respond with "DontCare" for irrelevant items or "DontKnow" for unclear ones, adapting the process.
  \item The loop iterates until all nodes are visited, ensuring comprehensive preference collection.
  \item At termination, the agent produces a global preference state for downstream generation tasks.
  \item Algorithm 1 formalizes the interaction workflow, providing a structured implementation.
    \end{itemize}

    \end{frame}
    
\end{document}
